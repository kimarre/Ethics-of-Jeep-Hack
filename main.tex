% Term paper proposal - Kim Arre
% CSC 300: Professional Responsibilities
% Dr. Clark Turner

\documentclass[12pt]{article}

\usepackage{setspace}
\usepackage{url}

% Multicolumn
\usepackage{multicol}
\setlength{\columnsep}{.75cm}

\usepackage{lipsum}

%%% PAGE DIMENSIONS
%\usepackage[bottom=2cm, left=2cm, right=2cm]{geometry} % to change the page dimensions
\usepackage[right=2.5cm, left=2.5cm, bottom=3cm]{geometry}
\geometry{letterpaper}

\begin{document}

%%%%%%%%%%%%%%%%%%%
%%% Title Page  %%%
%%%%%%%%%%%%%%%%%%%

\title{\vfill Term Paper Proposal: The Ethics of Jeep Hack Breaking DMCA While Exposing Security Risks} %\vfill gives us the black space at the top of the page
\author{
  Kim Arre \vspace{10pt} \\
  CSC 300: Professional Responsibilities  \vspace{10pt} \\
  Dr. Clark Turner \vspace{10pt} \\
}
\date{\today}

\maketitle

\vfill  %in combination with \newpage this forces the abstract to the bottom of the page
\begin{abstract}
In July 2015 a security researcher named Charlie Miller remotely hacked into a Jeep from 10 miles away, taking over basic physical features of the car including turning on the windshield wipers, steering the car and disabling control of the transmission. In doing so, he violated The Digital Millennium Copyright Act's (DMCA) section on Anti-Circumvention by defeating Jeep's Technological Protection Measures (TPM). Was it ethical for Miller to be in violation of the DMCA in order to unveil a major security vulnerability in Jeep cars? Some contend that in order to maintain the integrity of car manufacturer's market interests, the TPM should have been respected and left alone. However, without circumventing the car's TPM, 1.4 million cars wouldn't have been recalled for the security risk. Miller acted in the interest of the public good and should not have been in violation of the DMCA.
\end{abstract}

\thispagestyle{empty} %remove page number from title page, but still keep it as pg #1
\newpage

\tableofcontents
\newpage
\begin{multicols}{2}

%%%%%%%%%%%%%%%%%%%%
%%% Known Facts  %%%
%%%%%%%%%%%%%%%%%%%%
\section{Facts}

In July 2015, well known security researcher Charlie Miller and his research partner Chris Valasek successfully took control of an unaltered Jeep from miles away. \cite{wired}

Dr. Miller security background includes work with the National Security Agency (NSA) and  research on automotive security begun in 2013 when he showed that with direct access to a vehicle, he could control the physical systems of a 2010 Ford Escape and Toyota Prius. \cite{officialPaper} After unveiling his findings, Toyota released a statement that the attacks were only possible because they had physical access to the vehicles and that ``Our focus, and that of the entire auto industry, is to prevent hacking from a remote wireless device outside of the vehicle." \cite{originalHack}

When the automotive industry dismissed Miller after he displayed the possible threats, Miller followed up with further research on controlling an unaltered Jeep remotely from 10 miles away by taking advantage of the car's Uconnect system. \cite{officialPaper} This allowed him to be able to remotely control physical functions of the car including the car's windshield wipers, sound system, steering, and transmission. \cite{wired}  Only days after Andy Greenberg released an article on Wired, Miller's research on the topic resulted in Chrysler's recall of 1.4 million Jeep vehicles.\cite{recall} 

However, the very act of confessing that they hacked into cars to do automotive security research left the two researchers susceptible to Jeep pressing charges against them. \cite{brokeDMCA} Although in the end, Jeep did not press charges, Miller and Valasek violated The Digital Millennium Copyright Act's (DMCA) section on Anti-Circumvention. Section 1201(a.1.A) of the DMCA states: ``No person shall circumvent a technological measure that effectively controls access to a work protected under this title."\cite{DMCA}


%%%%%%%%%%%%%%%%%%%%%%%%%
%%% Research Question %%%
%%%%%%%%%%%%%%%%%%%%%%%%%
\section{Research question} 

Was it ethical for Charlie Miller to be to be in violation of the Digital Millennium Copyright Act for unveiling a major security vulnerability in Jeep cars?

%%%%%%%%%%%%%%%%%%%%%%%%%%%
%%% Social Implications %%%
%%%%%%%%%%%%%%%%%%%%%%%%%%%
\section{Social Implications} 
This case looks at whether or not the DMCA is being constructive in protecting the rights of the correct people. It's anti-circumvention law is widely criticized for making software secure through obscurity. The general logic of it is that if it is illegal to bypass well written and obscure security locks (TPMs), it doesn't matter what software lies beyond since it's the company's software and no one else is allowed to examine it for any reason.\cite{dictionary} This provides social issues for whether or not we should trust the car manufacturers, but suppresses the general public from figuring out if the product they're using actually is safe and does what we think it does. By risking punishment for security researchers looking into the code of products, it discourages potential security flaws to be unveiled and fixed, leaving it susceptible to being taken advantage of by someone from a foreign country who is not under the same laws.\cite{turner}

%%%%%%%%%%%%%%%%%%%%%%%%%%%%%%%%%%%%%%%%%%%%%%
%%% Extant Arguments from External Sources %%%
%%%%%%%%%%%%%%%%%%%%%%%%%%%%%%%%%%%%%%%%%%%%%%
\section{Arguments}

% --- Affirmative ---
\subsection{Arguments Affirmative}
\subsubsection{The software TPMs were put in place to protect Jeep's copyright and prevent their code from being stolen or sold to other car manufacturers}

In order for vehicle companies to stay competitive with the others in their field, they must use technological protection measures (TPMs) to ensure that their intellectual property is safe from being stolen or used against them in the market. 
     
\subsubsection{It is a disservice to the public good to risk the car's code landing in the hands of people with malicious intent}

By circumventing the TPM on cars, Miller put all cars affected at risk when he published the specifics on how he did it\cite{officialPaper} for potential wrong-doers to find. 

% --- Negative --- 
\subsection{Arguments Negative}

\subsubsection{Discouraging the research of vehicles prevents us from unveiling potential dishonesty from manufacturers}

As seen in the recent unveiling of Volkswagen's scandal, research was essential for uncovering Volkswagen's dishonesty regarding their Clean Diesel technology. \cite{vwScandal} Without this research, we wouldn't have discovered the true emission rates.

\subsubsection{Manufacturers should be doing more to ensure their vehicles are safe by allowing for more extensive tests}

When Toyota made the statement that they weren't concerned with security risks \cite{originalHack} before Miller proved them wrong, it's unclear if they genuinely believed their security was rock solid. There's a possibility that they hadn't done enough testing but assumed that security by obscurity would be enough\cite{brokeDMCA}. Car manufacturers should follow in the footsteps of Uber who hires researchers like Miller and Valasek to ensure the security of their vehicles\cite{uber}.


%%%%%%%%%%%%%%%%%%%%%%%%%%%
%%% Analytic principles %%%
%%%%%%%%%%%%%%%%%%%%%%%%%%%
\section{Analysis}

\subsection{Why the Software Engineering Code of Ethics is applicable to this problem}

The Software Engineering Code of Ethics states that software engineers ``are those who contribute by direct participation or by ... the analysis ... and testing of software systems."\cite{seCode} Did Charlie Miller ``directly participate" to the analysis of a ``software system"?

Hacking around a car's TPMs was Charlie Miller's form of direct participation. Additionally,  Miller and Valasek published a paper on how they hacked into the car\cite{officialPaper}, as well as gave a demonstration\cite{wired} and a DEF CON Hacking Conference talk on it\cite{youtube}. This makes them direct participants with the car as a ``software system'', and thus software engineers under the SE Code of Ethics and ``shall adhere to the code."\cite{seCode}

The SE Code's preamble also includes that ``... software engineers must commit themselves to making software engineering a beneficial and \underline{respected} profession.''

In Miller and Valasek's DEF CON presentation, one of their opening slides included the statement, ``please stop saying unhackable, you're going to look silly." The auto industry companies were acting in ways that were not consistent in keeping the software related to the cars when they said that their cars were "unhackable."

Charlie Miller's unveiling of the security vulnerabilities served to make more of an effort to ensure auto industry companies act in more respectable ways by challenging their claims on security. Therefore, Charlie Miller further qualifies as a software engineer, under which the SE Code applies.




%\subsection {SE Code section 6.05 (supported by 2.05)}

%SE Code section 6.05 states that software engineers are to ``Not \underline{promote their own interest} at the \underline{expense} of the \underline{profession, client} or employer"\cite{seCode}

%The definition of ``promote'' is ``to help or encourage to exist or flourish; further"\cite{dictionary} as well as the informal definition of ``promotion'' as ``to obtain (something) by cunning or trickery''\cite{dictionary}

%``expense'' in 

%The ``profession or clients'' in this case are the vehicle companies such as Jeep, Ford, and Toyota, who create and manufacture the cars Miller was testing on.

%Miller worked in his own interest when he bypassed Jeep's software security system. In this context, it was done at the expense of the security of the software Jeep uses for their cars. 



%maybe do this one instead?
\subsection {SE Code section 2.05}

SE Code section 2.05 states the software engineers are to ``Keep private any \underline{confidential} information gained in their \underline{professional work}, where such confidentiality is consistent with the \underline{public interest} and \underline{consistent with the law}.''

``Confidential'' means to speak or write about something only in strict privacy.\cite{dictionary}

"Professional work" refers to Charlie Miller's knowledge in security research, and his involvement in researching the security software behind the Jeep vehicle. 

The public interest in the context of keeping confidentiality safe is the public's general security concerns to trust that their cars won't randomly be hacked while in use. 

The consistency with the law for the Jeep case directly relates to the DMCA's anti-circumvention law.

\vspace{.5cm}\textbf{Substituted SE Code 2.05:}\vspace{.25cm}

Do not publicly speak or write about any information gained from security research on the Jeep car that could put the general public at risk of having their cars hacked while in use.





\subsection{SE Code sections 6.06 and 2.02}
SE Code 6.06 states that software engineers are to ``Obey all \underline{laws} governing their \underline{work}, unless, in \underline{exceptional circumstances}, such compliance is inconsistent with the \underline{public interest}."\cite{seCode}

The definition of a ``law" is ``any written or positive rule or collection of rules prescribed under the authority of the state or nation, as by the people in its constitution."\cite{dictionary} Under this tenet, the ``laws" being applied to this problem are the DMCA's anti-circumvention law. 





The meaning of ``work'' is ``exertion or effort directed to produce or accomplish something" \cite{dictionary}. Miller's intent to accomplish was proving to the automobile industry that their cars in fact were not unhackable. \cite{youtube}

The ``public interest" in this case would be considered as the general interest people have over their personal vehicles. This is stated as such in Miller and Valasek's published paper on the hack: ``Car security research is interesting for a general audience because most people have cars and understand the inherent dangers of an attacker gaining control of their vehicle." \cite{officialPaper}

The definition of ``exceptional" is "forming an exception or rare instance; unusual." \cite{dictionary} Typically, laws are made to protect the people under which they govern. Therefore, an example of an ``unusual'' circumstance would be one in which the law is doing the opposite and preventing them from being protected.

\vspace{.5cm}\textbf{Substituted SE Code 6.06:}\vspace{.25cm}

Obey the DMCA's anti-circumvention law, unless, an unusual circumstance occurs that the leaves public at risk of unprotected.

In support, SE Code section 2.02 states software engineers should ``Not knowingly use software that is obtained or retained either illegally or unethically'' \cite{seCode} which is exactly what Charlie Miller did. The DMCA renders Miller's acquisition of security research findings illegal.  Since he needed to bypass the car's TPMs in order to conduct his research, he was not obeying the law. Miller did not follow his ethical duty as a software engineer when he knowingly broke DMCA Section 1201.

However, because Miller was conducting research on the security of consumer vehicles, Miller was acting consistently with the public interest of safety.




\subsection{SE Code section 6.08}

Section 6.08 of the SE Code states the Software Engineers shall ``Take \underline{responsibility} for \underline{detecting}, correcting, and \underline{reporting} errors in software and associated documents on which they work."\cite{seCode}

The definition of ``responsibility'' is ``the state ... of being responsible, answerable, or accountable for something within one's power ...''\cite{dictionary}
Charlie Miller's extensive knowledge of software security systems as described by his former experience with the NSA and Twitter\cite{linkedin}, security research is well 'within his power,' compared to anyone with less experience, who wouldn't be considered to have this kind of research within their power or skill set.

``To detect'' is defined as ``to discover or catch (a person) in the performance of some act,''\cite{dictionary} or ``to discover the existence of''\cite{dictionary}. 

Under this section of the SE code, Miller dedicated months of research on detecting security vulnerabilities with Jeep, Ford and Prius vehicles.

He reported these errors both to the public and to the car manufacturers themselves so the problems could be remedied. 

%%%%%%%%%%%%%%%%%%%%%%%%%%%%%%%%%%%%%%%
%%% Abstract your Expected Analysis %%%
%%%%%%%%%%%%%%%%%%%%%%%%%%%%%%%%%%%%%%%
\section{Conclusion}

No, because Charlie Miller behaved ethically under the SE Code of Ethics, it was not ethical for Miller to be at risk of being charged for circumventing Jeep's security system. Overall, Miller acted in the interest of the public good. The potential damage that could have resulted to Jeep owners far exceeded Jeep's copyright concerns.

\end{multicols}

%cite all the references from the tex you haven't explicitly cited
\nocite{*}

\bibliographystyle{IEEEannot}

\newpage
\bibliography{proposal}
\end{document}